%!TEX TS-program=xelatex
\documentclass[xetex]{beamer}
% 将上面这一行修改成下面这个样子,可以创建适合于发布的版本,这去除了所有的动画
% \documentclass[xetex, handout]{beamer}

% 规范注意:

% 使用正确的主题(beamer主题、文字字体)
% 使用正确的title信息(title、subtitle、author、date)
% 合理使用frame 和 standout frame
% 块(block、exampleblock、alertblock)以及文字段落内部的\alert的使用
% 使用图片需要使用\begin{figure}...\end{figure}并附带\caption和\label信息
% 使用enumerate和itemize组织你的点
% 使用section给你的幻灯片分部分
% 公式,文字段落内嵌公式和单独的公式块的使用
% \DeclareMathOperator的使用,以及学会在网上查找你不知道怎么输入的数学符号
% 动画的使用
% 讲课录制使用什么版本的文档;对外发布使用什么版本的文档(handout)

\usefonttheme{professionalfonts}

\usepackage[UTF8]{ctex}
\usepackage{hyperref}
\usepackage{unicode-math}
\usepackage{amsmath, amssymb}
\usepackage{graphicx, wrapfig}
%\usepackage{XITSMath}
\usepackage{nopageno}

\DeclareMathOperator{\argmax}{argmax}

\usetheme[block=fill]{metropolis}
%\setmainfont{XITS} 
%\setmathfont{xits-math.otf}

\setmainfont{XITS}
[    Extension = .otf,
   UprightFont = *-Regular,
      BoldFont = *-Bold,
    ItalicFont = *-Italic,
BoldItalicFont = *-BoldItalic,
]
\setmathfont{XITSMath-Regular}
[    Extension = .otf,
      BoldFont = XITSMath-Bold,
]

\title{含参变量的积分}
\subtitle{含参变量的正常积分}
\author{数学分析MOOC小组}
\date{2020年8月20日}


\begin{document}

\frame{\maketitle}

\begin{frame}
    \frametitle{含参变量积分连续性}
    \begin{block}{定理19.1}
        设$f(x,y)$在矩形区域$[a,b]\times[c,d]$上连续,
        则$I(x)=\int_{c}^d{f(x,y)dy}$在$x\in[a,b]$上连续
    \end{block}
    $\forall x \in [a,b] \Rightarrow \lim\limits_{x \to x_0}I(x)=I(x_0)$
    即$\lim\limits_{x \to x_0}\int_{c}^d{f(x,y)dy}
    =\int_{c}^d{f(x_0,y)dy}
    =\int_{c}^d{\lim\limits_{x \to x_0}f(x,y)dy}$
    
    \alert{积分和极限运算可以交换顺序}
    \begin{block}{例题}
        $\lim\limits_{a \to 0}\int_{-1}^{1}{\sqrt{x^2+a^2}dx}$
        
        解:由于$\sqrt{x^2+a^2}$在区间$[-1,1]$上连续
        $\newline\lim\limits_{a \to 0}\int_{-1}^{1}{\sqrt{x^2+a^2}dx}
        =\int_{-1}^{1}{\lim\limits_{a \to 0}\sqrt{x^2+a^2}dx}
        =\int_{-1}^{1}{\vert x \vert dx}=1$

    \end{block}


\end{frame}

\begin{frame}
    \frametitle{积分下求导}
    \begin{block}{定理19.2}
        设$f(x,y)$和$f_x^{'}(x,y)$在矩形区域$[a,b]\times[c,d]$上连续,
        则$I(x)=\int_{c}^d{f(x,y)dy}$在$[a,b]$上有连续的导函数
        且$I^{'}(x)=\int_{c}^d{f_x^{'}(x,y)dy} \Leftrightarrow \frac{d}{dx}\int_{c}^d{f(x,y)dy=\int_{c}^d{\frac{\partial}{\partial x}f(x,y)dy}}$
    \end{block}

    \alert{积分和求导运算可以交换顺序,积分限为常数}
    $\newline$

    \begin{block}{例题}
        $$I=\int_{0}^{1}{\frac{\ln(1+x)}{1+x^2}dx}$$
    \end{block}
\end{frame}

\begin{frame}

    解:本题直接进行定积分求解求不出来,需引入含参变量积分
    \begin{enumerate}
        \item 首先引入参变量$a$
        $I(a)=\int_{0}^{1}{\frac{\ln(1+ax)}{1+x^2}dx}\ \ a\in[0,1]$
        $\ \ \ f(x,a)=\frac{\ln(1+ax)}{1+x^2}$
        $f_a^{'}(x,a)=\frac{x}{(1+x^2)(1+ax)}=\frac{1}{1+a^2}(\frac{a+x}{1+x^2}-\frac{a}{1+ax})$
        $\newline$
        \item 由于它们都在$[0,1]\times[0,1]$上连续,则
        $I^{'}(a)=\int_{0}^{1}{f_a^{'}(x,a)dx}
        =\int_{0}^{1}{\frac{1}{1+a^2}(\frac{a+x}{1+x^2}-\frac{a}{1+ax})dx}
        =\frac{1}{1+a^2}[a\arctan{x}+\frac{1}{2}\ln{1+x^2}-\ln(1+ax)]|_{0}^{1}
        =\frac{1}{1+a^2}[\frac{\pi}{4}a+\frac{1}{2}\ln{2}-\ln(1+a)]$
        $\newline$
        \item 显然$I(0)=0$,则$I(1)=I(1)-I(0)=\int_{0}^{1}{I^{'}(a)da}
        =\int_{0}^{1}{\frac{1}{1+a^2}[\frac{\pi}{4}a+\frac{1}{2}\ln{2}-\ln(1+a)]}
        =[\frac{\pi}{8}\ln{1+a^2}+\frac{1}{2}\ln{2}\arctan{a}]|_0^{1}-\int_{0}^{1}{\frac{\ln(1+a)}{1+a^2}da}
        I=I(1)=\frac{\pi}{8}\ln{2}$
    \end{enumerate}

\end{frame}

\begin{frame}
    \frametitle{多元含参变量积分求导}
    \begin{block}{定理19.3}
        设$f(x,y)$在矩形区域$[a,b]\times[c,d]$上连续
        $I(x,u)=\int_c^{u}{f(x,y)dy}$
        \begin{enumerate}
            \item $I(x,u)$在矩形区域$[a,b]\times[c,d]$上连续
            \item 若$f^{'}_x(x,y)$在$[a,b]\times[c,d]$上连续,则$I(x,u)$在$[a,b]\times[c,d]$上对各变元有连续偏导数
        \end{enumerate}
    \end{block}
    $$\frac{\partial I}{\partial u}=f(x,u) \ \ \ \frac{\partial I}{\partial x}=\int_c^{u}{f^{'}_x(x,u)}$$
\end{frame}

\begin{frame}
    \frametitle{多元含参变量积分求导}
    \begin{block}{定理19.4}
        设$f(x,y)$在矩形区域$[a,b]\times[c,d]$上连续
        $c(x),d(x)$在$[a,b]$上连续,并且当$x\in [a,b]$有$c\le c(x),d(x)\le d$
        则$F(x)=\int_{c(x)}^{d(x)}{f(x,y)dy}$在$[a,b]$上连续
        
    \end{block}
    \begin{block}{证明}
        $I(x,u)=\int_c^{u}{f(x,y)dy}$
        $\newline F(x)=\int_{c}^{d(x)}{f(x,y)dy}-\int_{c}^{c(x)}{f(x,y)dy}=I(x,d(x))-I(x,c(x))$ 
    \end{block}
\end{frame}

\begin{frame}
    \frametitle{多元含参变量积分求导}
    \begin{block}{定理19.5}
        设$f(x,y)$和$f_x^{'}(x,y)$在矩形区域$[a,b]\times[c,d]$上连续
        $c(x),d(x)$在$[a,b]$上可导,且当$x\in [a,b]$有$c\le c(x),d(x)\le d$
        则$F(x)=\int_{c(x)}^{d(x)}{f(x,y)dy}$在$[a,b]$上可导
        且$F^{'}(x)=\int_{c(x)}^{d(x)}{f^{'}_x(x,y)dy}+f(x,d(x))d^{'}(x)-f(x,c(x))c^{'}(x)$
    \end{block}
    \begin{block}{证明}
        $F(x)=\int_{c(x)}^{d(x)}{f(x,y)dy}$
        令$u=d(x), v=c(x), I(x,u)=\int_c^{u}{f(x,y)dy}$
        则$F(x)=I(x,u)-I(v,x)$ 
        $F^{'}(x)=\frac{\partial I}{\partial x} + \frac{\partial I}{\partial u} \frac{\partial u}{\partial x} 
        - \frac{\partial I}{\partial x} + \frac{\partial I}{\partial v} \frac{\partial v}{\partial x}$
        
        根据定理19.3可知$\frac{\partial I}{\partial u}=f(x,u) \ \ \ \frac{\partial I}{\partial x}=\int_c^{u}{f^{'}_x(x,u)}$
        
        $\int_{c(x)}^{d(x)}{f^{'}_x(x,y)dy}+f(x,d(x))d^{'}(x)-f(x,c(x))c^{'}(x)$
    \end{block}
\end{frame}

\begin{frame}
    \frametitle{例题}
    \begin{block}{例题}
        $F(x)=\int_{0}^{x}{[\int_{t^2}^{x^2}f(t,s)ds]dt}\ \ \ $求$F^{'}(x)$
    \end{block}
    \begin{block}{解}
        令$I(t, x^2, t^2)=\int_{t^2}^{x^2}{f(t,s)ds} \ \ \ \frac{\partial I}{\partial x^2}=f(t,x^2)\ $
        
        $F(x)=\int_{0}^{x}{I(t,x^2,t^2)dt}$
        
        令$d(x)=x, c(x)=0$
        对$F(x)$求导数
        $$\newline F^{'}(x)=\int_{c(x)}^{d(x)}{\frac{\partial I}{\partial x}dt}+I(d(x),x^2,d(x)^2)d^{'}(x)-I(c(x),x^2,c(x)^2)c^{'}(x)$$
        \Large $=\int_{0}^{x}{\frac{\partial I}{\partial x^2}\frac{\partial x^2}{\partial x}dt}+I(x,x^2,x^2)$
        \linespread{1.7}\Large $\newline=\int_{0}^{x}{2xf(t,x^2)dt}+\int_{x^2}^{x^2}{f(x,s)ds}
        =\int_{0}^{x}{2xf(t,x^2)dt}$

    \end{block}
\end{frame}


\begin{frame}
    \frametitle{积分交换次序}
    \begin{block}{定理19.6}
        设$f(x,y)$在矩形区域$[a,b]\times[c,d]$上连续
        $I(x)=\int_{c}^{d}{f(x,y)dy}$
        则$\int_{a}^{b}{I(x)dx}=\int_{c}^{d}{dy}\int_{a}^{b}{f(x,y)dx}$
        即$\int_{a}^{b}{dx}\int_{c}^{d}{f(x,y)dy}=\int_{c}^{d}{dy}\int_{a}^{b}{f(x,y)dx}$
    \end{block}
    \begin{block}{例题}
        $\int_{0}^{1}{\frac{x^b-x^a}{\ln{x}}dx}$
    \end{block}

    \begin{block}{解}

        将$\frac{x^b-x^a}{\ln{x}}$看作积分$\frac{x^b-x^a}{\ln{x}}=\int_a^b{x^ydy}$
        $\newline\int_{0}^{1}{\frac{x^b-x^a}{\ln{x}}dx}=\int_{0}^{1}{dx}{\int_a^b{x^ydy}}
        =\int_{a}^{b}{dy}{\int_0^1{x^ydx}}
        \newline=\int_a^b{\frac{1}{1+y}dy}=\ln{(1+y)}|^b_a=\ln{\frac{1+b}{1+a}}$

    \end{block}
\end{frame}



\begin{frame}
    \frametitle{总结}
    \begin{enumerate}
        \item 含参变量积分
        \begin{itemize}
            \item 将含参变量积分视为函数
            \item 积分和极限运算交换顺序
        \end{itemize}
        \item 含参变量积分导数
        \begin{itemize}
            \item 积分限为常数,一元函数导数
            \item 积分限为变量,多元函数偏导数
        \end{itemize}
        \item 积分交换次序
    \end{enumerate}

\end{frame}

\begin{frame}[standout]
    \Huge 感谢观看!

\end{frame}

\end{document}