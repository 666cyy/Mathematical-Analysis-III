%!TEX TS-program=xelatex
\documentclass[xetex]{beamer}
% 将上面这一行修改成下面这个样子,可以创建适合于发布的版本,这去除了所有的动画
% \documentclass[xetex, handout]{beamer}

% 规范注意:

% 使用正确的主题(beamer主题、文字字体)
% 使用正确的title信息(title、subtitle、author、date)
% 合理使用frame 和 standout frame
% 块(block、exampleblock、alertblock)以及文字段落内部的\alert的使用
% 使用图片需要使用\begin{figure}...\end{figure}并附带\caption和\label信息
% 使用enumerate和itemize组织你的点
% 使用section给你的幻灯片分部分
% 公式,文字段落内嵌公式和单独的公式块的使用
% \DeclareMathOperator的使用,以及学会在网上查找你不知道怎么输入的数学符号
% 动画的使用
% 讲课录制使用什么版本的文档;对外发布使用什么版本的文档(handout)

\usefonttheme{professionalfonts}

\usepackage[UTF8]{ctex}
\usepackage{hyperref}
\usepackage{unicode-math}
\usepackage{amsmath, amssymb}
\usepackage{graphicx, wrapfig}
%\usepackage{XITSMath}
\usepackage{nopageno}

\DeclareMathOperator{\argmax}{argmax}

\usetheme[block=fill]{metropolis}
%\setmainfont{XITS} 
%\setmathfont{xits-math.otf}

\setmainfont{XITS}
[    Extension = .otf,
   UprightFont = *-Regular,
      BoldFont = *-Bold,
    ItalicFont = *-Italic,
BoldItalicFont = *-BoldItalic,
]
\setmathfont{XITSMath-Regular}
[    Extension = .otf,
      BoldFont = XITSMath-Bold,
]

\title{含参变量的积分}
\subtitle{含参变量的广义积分}
\author{数学分析MOOC小组}
\date{2020年8月21日}

\begin{document}

\frame{\maketitle}

\begin{frame}
    \frametitle{一致收敛}
    \begin{block}{含参变量广义积分一致收敛定义}
        设$f(x,y)$定义在$[a,b]\times[c,+\infty]$,且对$\forall x\in[a,b]$
        无穷积分$I(x)=\int_{c}^{+\infty}{f(x,y)dy}$收敛,
        若对于$\forall \epsilon > 0, \exists A_0 > c$,当$A>A_0$时有
        $\vert \int_c^A{f(x,y)dy}-I(x)\vert < \epsilon$或$\vert \int_A^{+\infty}{f(x,y)dy}\vert < \epsilon$
        对$\forall x \in [a,b]$都成立,则称含参变量广义积分$\int_c^{+\infty}{f(x,y)dy}$在$[a,b]$一致收敛
    \end{block}

    \begin{block}{例题}
        证明:$\int_{0}^{+\infty}{xe^{-xy}dy}$在区间$[a,+\infty)\ (a>0)$一致收敛
        $\int_{A}^{+\infty}{xe^{-xy}dy}=-e^{-xy}|^{+\infty}_{A}=e^{-Ax}\le e^{-Aa}<\epsilon \Leftrightarrow A > \frac{-ln{\epsilon}}{a} = A_0$
    \end{block}


\end{frame}

\begin{frame}
    \frametitle{一致收敛}
    含参变量广义积分$f(x,y)$定义在$[a,b]\times[c,+\infty]$上,且对$\forall x\in[a,b]$
    无穷积分$\int_{c}^{+\infty}{f(x,y)dy}$一致收敛的充要条件
    \begin{block}{柯西准则}
        对$\forall \epsilon>0,\exists A_0(A_0>0)>c$,当$A',A''>A_0$时,对于$\forall x \in [a,b]$有
        $\int_{A'}^{A''}{f(x,y)dy}<\epsilon$
    \end{block}
    \begin{block}{M判别法}
    \begin{enumerate}
        \item 设存在函数$M(y)$与常数$B>c$使得当$y\ge B$与$x\in [a,b]$时,有$\vert f(x,y)\vert\le M(y)$
        \item 广义积分$\int_{c}^{+\infty}{M(y)}$收敛
    \end{enumerate}
    \end{block}

\end{frame}

\begin{frame}
    \frametitle{一致收敛}
    含参变量广义积分$\int_c^{+\infty}{f(x,y)g(x,y)dy}$在$x\in[a,b]$上一致收敛充要条件
    \begin{block}{狄利克雷判别法}
        \begin{enumerate}
            \item 含参变量正常积分$\int_c^A{f(x,y)dy}$在$A\ge c$与$x\in [a,b]$有界,即存在$M>0$对$\forall A>c$及$\forall x\in [a,b]$有$\vert \int_c^A{f(x,y)dy}\vert \le M$
            \item 对每个固定的$x\in[a,b]$,函数$g(x,y)$关于y是单调的,且当$y \to +\infty$时,$g(x,y)$在$x\in[a,b]$一致的趋近于0
        \end{enumerate}
    \end{block}   
    \begin{block}{阿贝尔判别法}
        \begin{enumerate}
            \item $\int_c^{+\infty}{f(x,y)dy}$在$x\in[a,b]$上一致收敛
            \item 对每个固定的$x\in[a,b]$,函数$g(x,y)$关于y是单调的,且$g(x,y)$在$x\in[a,b],y\ge c$有界
        \end{enumerate}
    \end{block}   
\end{frame}

\begin{frame}
    \frametitle{例题}
    证明下列积分一致收敛
    \begin{block}{1}
        $\int_0^{+\infty}{\frac{cos(xy)}{x^2+y^2}dy}\ (x\ge a >0)$
        
        证明:当$(x\ge a >0)$时,$\forall y \in [0, +\infty]$有
        $$\vert \frac{cos(xy)}{x^2+y^2}\vert \le \frac{1}{x^2+y^2}\le \frac{1}{a^2+y^2}$$
        $$\int_0^{+\infty}{\frac{1}{a^2+y^2}dy}=\frac{1}{a}\arctan {\frac{y}{a}}|^{+\infty}_0=\frac{\pi}{2a}$$
        根据M判别法可知上述广义积分一致收敛
    \end{block}
    

\end{frame}


\begin{frame}
    \frametitle{例题}
    \begin{block}{2}
        $\int_{1}^{+\infty}{\frac{e^{-xy}\cos y}{y^p}dp}(p>0)\ x\in[0,+\infty)$
        
        证明:将被积函数分为两部分$f(x,y)=\cos y\ \ \ g(x,y)=\frac{e^{-xy}}{y^p}$
        \begin{enumerate}
            \item $\vert \int_1^A{cos(y)}\vert = \vert \sin A- \sin 1 \vert \le 2 $
            \item $\frac{e^{-xy}}{y^p}$在$y\ge 1$时,关于$y$单调下降,且当$y \to +\infty$关于$x$一致趋向于0
            根据狄利克雷判别法可知上述广义积分一致收敛
        \end{enumerate}
    \end{block}
\end{frame}

\begin{frame}
    \frametitle{含参变量广义积分连续性}
    \begin{block}{定理19.1}
        设$f(x,y)$在矩形区域$[a,b]\times[c,+\infty)$上连续,
        若含参变量广义积分$I(x)=\int_{c}^{+\infty}{f(x,y)dy}$在$x\in[a,b]$上一致收敛,
        则$I(x)$在$x\in[a,b]$上连续
    \end{block}
    $\forall x \in [a,b] \Rightarrow \lim\limits_{x \to x_0}I(x)=I(x_0)$
    即$\lim\limits_{x \to x_0}\int_{c}^{+\infty}{f(x,y)dy}
    =\int_{c}^{+\infty}{f(x_0,y)dy}
    =\int_{c}^{+\infty}{\lim\limits_{x \to x_0}f(x,y)dy}$
    
    \alert{积分和极限运算可以交换顺序}

\end{frame}


\begin{frame}
    
    \begin{block}{例题}
        判断函数$F(x)=\int_{0}^{+\infty}{\frac{y^2}{1+y^x}dy}$在$x \in[3,+\infty)$上的连续性
        $\forall x_0>3, \exists 3<b<x_0$当$y\ge 1$,$\forall x \in [b,+\infty)$,有
        $$\frac{y^2}{1+y^x}=\frac{1}{y^{-2}+y^{x-2}}<\frac{1}{y^{x-2}}\le \frac{1}{y^{b-2}}$$
        同时$\int_1^{+\infty}{\frac{1}{y^{b-2}}}$收敛,根据M判别法$\int_{0}^{+\infty}{\frac{y^2}{1+y^x}dy}$在$[b,+\infty)$一致收敛,
        则最终$F(x)$在$[3,\infty)$上连续
    \end{block}

\end{frame}



\begin{frame}
    \frametitle{含参广义积分求导}
    \begin{block}{定理19.2}
        设$f(x,y)$和$f_x^{'}(x,y)$在矩形区域$[a,b]\times[c,+\infty)$上连续,
        若含参变量广义积分$I(x)=\int_{c}^{+\infty}{f(x,y)dy}$在$[a,b]$上收敛,
        若含参变量广义积分$I(x)=\int_{c}^{+\infty}{f^{'}_x(x,y)dy}$在$[a,b]$上一致收敛,

        则$I(x)=\int_{c}^{+\infty}{f(x,y)dy}$在$[a,b]$上可导
        且$I^{'}(x)=\int_{c}^{+\infty}{f_x^{'}(x,y)dy} \Leftrightarrow \frac{d}{dx}\int_{c}^{+\infty}{f(x,y)dy=\int_{c}^{+\infty}{\frac{\partial}{\partial x}f(x,y)dy}}$
    \end{block}

    \alert{积分和求导运算可以交换顺序}
    $\newline$

    \begin{block}{例题}
        $$I=\int_{0}^{+\infty}{\frac{1-e^{-t}}{t}\cos tdt}$$
    \end{block}
\end{frame}

\begin{frame}
    解:本题直接进行定积分求解求不出来,需引入含参变量积分
    \begin{enumerate}
        \item 首先引入参变量$a(>0)$
        $I(a)=\int_{0}^{+\infty}{\frac{1-e^{-at}}{t}\cos tdt}$
        
        对$\forall a > 0, \exists \ b>0:b<a$
        
        使得$\frac{1-e^{-at}}{t}\cos t$和$\frac{\partial}{\partial a}\frac{1-e^{-at}}{t}\cos t = e^{-at}\cos t$均在$[b,+\infty)\times[0,+\infty)$上连续

        对$1-e^{-at} \cos t$单调有界,对于$\int_0^{+\infty}{\frac{\cot t}{t}dt}$收敛,根据广义积分的阿贝尔判别法,其收敛

        对$\forall a \in [b,+\infty)$存在$\vert e^{-at}\cos t \le e^{-bt}\vert$,同时$\int_{0}^{+\infty}{e^{-bt}dt}$收敛,则$\int_0^{+\infty}{e^{-at}\cos t dt}$一致收敛
        \item 求导数
        对$\forall a \in [b,+\infty)$有$I^{'}(a)=\int_{0}^{+\infty}{e^{-at}\cos t dt}=\frac{-a\cos t+\sin t}{a^2+1}e^{-at}|^{+\infty}_0=\frac{a}{a^2+1}$
        则$I(a)=\frac{1}{2}\ln{1+a^2}+c$
        \item 根据$I(0)=0 \Rightarrow c=0$,$I(1)=\frac{1}{2}\ln {2}$
    \end{enumerate}
\end{frame}


\begin{frame}
    \frametitle{积分交换次序}
    \begin{block}{定理19.6}
        设$f(x,y)$在矩形区域$[a,b]\times[c,+\infty)$上连续,
        若含参变量广义积分$I(x)=\int_{c}^{+\infty}{f(x,y)dy}$在$x\in[a,b]$上一致收敛,
        则$\int_{a}^{b}{I(x)dx}=\int_{c}^{+\infty}{dy}\int_{a}^{b}{f(x,y)dx}$
        即$\int_{a}^{b}{dx}\int_{c}^{+\infty}{f(x,y)dy}=\int_{c}^{+\infty}{dy}\int_{a}^{b}{f(x,y)dx}$
    \end{block}
    $\newline$
    \begin{block}{例题}
        $\int_{0}^{+\infty}{\frac{e^{-ax^2}-e^{-bx^2}}{x}dx} (a>0, b>0)$
    \end{block}
        
\end{frame}


\begin{frame}
    解:令A>0
    \begin{enumerate}
        \item $\int_0^A{e^{-ax^2}-e^{-bx^2}}\le \int_0^A{e^{-ax^2}+e^{-bx^2}}\le \int_0^A{\frac{1}{ax^2+1}+\frac{1}{bx^2+1}}\le \frac{1}{\sqrt{a}}\arctan{\sqrt{a}A}+\frac{1}{\sqrt{b}}\arctan{\sqrt{b}A}$
        \item $\frac{1}{x}$单调,且$x\to +\infty$时趋近于0
    \end{enumerate}
    根据广义积分的狄利克雷判别法可以得知,此广义积分一致收敛

    将$\frac{e^{-ax^2}-e^{-bx^2}}{x}$看作$-x\int_a^b{e^{-tx^2}dt}$
    $\newline\int_{0}^{+\infty}{\frac{e^{-ax^2}-e^{-bx^2}}{x}dx}=-\int_{0}^{+\infty}{xdx}{\int_b^a{e^{-tx^2}dt}}
    =\int_{a}^{b}{dt}{\int_0^{+\infty}{xe^{-tx^2}dt}}
    \newline -\int_{a}^{b}{\frac{1}{2t}dt}{\int_0^{+\infty}{e^{-tx^2}d(-tx^2)}}
    =-\int_{a}^{b}{\frac{1}{2t}e^{-tx^2}|^{+\infty}_0 dt}
    =\int_{a}^{b}{\frac{1}{2t}dt}=\frac{1}{2}\ln \frac{b}{a}$
        
\end{frame}

\begin{frame}
    \frametitle{积分交换次序}
    \begin{block}{迪尼定理}
        含参变量广义积分$f(x,y)$定义在$[a,+\infty)\times[c,+\infty)$上连续,非负。
        若$\int_{a}^{+\infty}{f(x,y)dy}$在$[c,+\infty)$收敛,且作为$y$的函数在$[c,d]$上连续,
        则$\int_{a}^{+\infty}{f(x,y)dy}$在$[c,+\infty)$是一致收敛
    \end{block}
    \begin{block}{积分交换次序}
        含参变量广义积分$f(x,y)$定义在$[a,+\infty)\times[c,+\infty)$上连续,非负。
        $\int_{a}^{+\infty}{f(x,y)dx}$,$\int_{c}^{+\infty}{f(x,y)dy}$都收敛,且分别在$[a,+\infty)$和$[c,+\infty)$上连续,
        若$\int_{a}^{+\infty}{dy}\int_{c}^{+\infty}{f(x,y)dx}$,$\int_{a}^{+\infty}{dx}\int_{c}^{+\infty}{f(x,y)dy}$
        中有一个存在另一个也存在,且相等
    \end{block}
\end{frame}

\begin{frame}
    \frametitle{例题}
    \begin{block}{例}
        $I=\int_0^{+\infty}{du}\int_0^{+\infty}{ue^{-(1+t^2)u^2}dt}$
    \end{block}
    \begin{block}{解}
        $\int_0^{+\infty}{ue^{-(1+t^2)u^2}du}=\frac{1}{2}\frac{1}{1+t^2}$
        $\newline \int_0^{+\infty}{ue^{-(1+t^2)u^2}dt}=\frac{1}{e^{u^2}}\int_0^{+\infty}{\frac{1}{e^{u^2t^2}}dt} \le \frac{1}{e^{u^2}}\int_0^{+\infty}{\frac{1}{u^2t^2+1}dt}$
        
        根据积分交换原则
        $\newline I=\int_0^{+\infty}{du}\int_0^{+\infty}{ue^{-(1+t^2)u^2}dt}
        \newline=\frac{1}{2}\int_0^{+\infty}{\frac{1}{1+t^2}e^{-(1+t^2)u^2}|^{0}_{-\infty}dt}
        \newline=\frac{1}{2}\int_0^{+\infty}{\frac{1}{1+t^2}dt}
        \newline=\frac{1}{2}\arctan t|^{+\infty}_0=\frac{\pi}{4}$
    \end{block}
\end{frame}

\begin{frame}
    \frametitle{总结}
    \begin{enumerate}
        \item 一致收敛
        \item 含参变量广义积分
        \begin{itemize}
            \item 将含参变量广义积分视为函数
            \item 积分和极限运算交换顺序
        \end{itemize}
        \item 含参广义变量积分导数
        
        \item 积分交换次序
        \begin{itemize}
            \item 广义积分和正常积分交换
            \item 广义积分和广义积分交换
        \end{itemize}
    \end{enumerate}

\end{frame}

\begin{frame}[standout]
    \Huge 感谢观看!

\end{frame}

\end{document}
