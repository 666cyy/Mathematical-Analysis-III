%!TEX TS-program=xelatex
\documentclass[xetex]{beamer}
% 将上面这一行修改成下面这个样子,可以创建适合于发布的版本,这去除了所有的动画
% \documentclass[xetex, handout]{beamer}

% 规范注意:

% 使用正确的主题(beamer主题、文字字体)
% 使用正确的title信息(title、subtitle、author、date)
% 合理使用frame 和 standout frame
% 块(block、exampleblock、alertblock)以及文字段落内部的\alert的使用
% 使用图片需要使用\begin{figure}...\end{figure}并附带\caption和\label信息
% 使用enumerate和itemize组织你的点
% 使用section给你的幻灯片分部分
% 公式,文字段落内嵌公式和单独的公式块的使用
% \DeclareMathOperator的使用,以及学会在网上查找你不知道怎么输入的数学符号
% 动画的使用
% 讲课录制使用什么版本的文档;对外发布使用什么版本的文档(handout)

\usefonttheme{professionalfonts}

\usepackage[UTF8]{ctex}
\usepackage{hyperref}
\usepackage{unicode-math}
\usepackage{amsmath, amssymb}
\usepackage{graphicx, wrapfig}

\usepackage{nopageno}

\DeclareMathOperator{\argmax}{argmax}

\usetheme[block=fill]{metropolis}

\setmathfont{XITS Math}

\title{多元函数的极限与连续性}
\subtitle{多元函数的极限与连续性}
\author{数学分析MOOC小组 }
\date{}

\begin{document}

\frame{\maketitle}

\begin{frame}
    \frametitle{概览}
	
    \begin{enumerate}
        \item 知识回顾
			\begin{enumerate}
           \item 全面极限
			\item 累次极限
			\item 连续性 
        	\end{enumerate}
        \item 习题讲解
        
    \end{enumerate}


\end{frame}

\section{知识回顾}
\begin{frame}
    \frametitle{全面极限} 
  设$f$在$P_{0}(x_{0},y_{0})$的某个空心邻域有定义,$A$是一个确定的数。如果对于$\forall \epsilon >0,\,\exists \delta>0$使得当$0<r(P,P_{0})<\delta$时,有$|f(P)-A|<\epsilon$,那么称$A$是二元函数$f$当$P\to P_{0}$时的极限,记为\\ $\displaystyle\lim_{x \to x_{0}\atop y \to y_{0}}f(x,y)=A$,或$\displaystyle\lim_{P \to P_{0}}f(P)=A$
\\解题中,下面表述\alert{更为方便}:\\
$\displaystyle\lim_{x \to x_{0}\atop y \to y_{0}}f(x,y)=A$,\\当且仅当对于$\forall\epsilon >0,\,\exists \delta>0$,只要$0<\sqrt{(x-x_{0})^2 + (y-y_{0})^2}<\delta$,就有$|f(P)-A|<\epsilon$
	
	
\end{frame}

\begin{frame}
    \frametitle{累次极限} 
   
若$\displaystyle\lim_{x \to x_{0}}\lim_{y \to y_{0}}f(x,y)$或$\displaystyle\lim_{y \to y_{0}}\lim_{x \to x_{0}}f(x,y)$存在,则称它们为累次极限。
	 
\end{frame}
 

\begin{frame}
    \frametitle{连续性}  
\textbf{二元函数的连续性}\qquad 设函数$f$在点$P_{0}$的某个邻域有定义,若$\displaystyle\lim_{P \to P_{0}}f(P)=f(P_{0})$,则称$f(P)$在点$P_{0}$连续。\\
\textbf{有界闭区域上连续函数的性质}:

    \begin{enumerate}
	\item[(1)]\textbf{有界性定理}\qquad 若$f(P)$在有界闭集$E$上连续,则$f(P)$在$E$上有界。
\item[(2)]\textbf{最值定理}\qquad 若$f(P)$在有界闭集$E$上连续,则$f(P)$在$E$上达到最大值和最小值。
\item[(3)]\textbf{一致连续性定理}\qquad 若$f(P)$在有界闭集$E$上连续,则$f(P)$在$E$上一致连续。
\item[(4)]\textbf{介值定理}\qquad $\mbox{若}f(P)\mbox{在区域}G\mbox{连续,}P_{1},P_{2}\in G$,$f(P_{1})<f(P_{2})$,则对于任意的$c:f(P_{1})<c<f(P_{2})$,存在$P_{0}\in G$,使得$f(P_{0})=c$。
    \end{enumerate}
\end{frame}

\section{习题讲解}
\begin{frame}
    \frametitle{习题2.(5)}
	 $\mbox{求}\displaystyle\lim_{x \to 0\atop y \to 0}x^2y^2\ln(x^2+y^2)$\\
   $\mbox{解:}\displaystyle\lim_{x \to 0\atop y \to 0}x^2y^2\ln(x^2+y^2)=\displaystyle\lim_{x \to 0\atop y \to 0}\frac{x^2y^2}{x^2+y^2}*\displaystyle\lim_{x \to 0\atop y \to 0}(x^2+y^2)\ln(x^2+y^2)$\\
$\mbox{一方面}0\le \frac{x^2y^2}{x^2+y^2}\le\frac{x^2y^2}{2|xy|}=\frac{|xy|}{2}\mbox{,而}\displaystyle\lim_{x \to 0\atop y \to 0}\frac{|xy|}{2}= 0$,由夹逼定理知$ \displaystyle\lim_{x \to 0\atop y \to 0}\frac{x^2y^2}{x^2+y^2}=0$;\\
$\mbox{另一方面,}\displaystyle\lim_{x \to 0\atop y \to 0}(x^2+y^2)\ln(x^2+y^2)=\displaystyle\lim_{t\to 0}t\ln t=0\mbox{,所以原式=0}$
\end{frame}

\begin{frame}
    \frametitle{习题8} 
   8.若$f(x,y)$在某区域$G$内对变量$x$连续,对变量$y$满足利普希茨条件,即对任意$(x,y')\in G$和$(x,y'')\in G$,有$|f(x,y')-f(x,y'')|\le L|y'-y''|$,其中$L$为常数,求证$f(x,y)$在$G$内连续。\\
	解:对于$\forall (x_{0},y_{0})\in G$,由于$f(x,y)$在$G$内对$x$连续,则对于$\forall \epsilon >0,\exists \delta'>0$,当$(x,y_{0})\in G$,且$|x-x_{0}|<\delta'$时,有$|f(x,y_{0})-f(x_{0},y_{0})|<\frac{\epsilon}{2}$。另一方面,$f(x,y)$在$G$内对$y$满足利普希茨条件,则对于$\forall (x,y)\in G$和$(x,y_{0})\in G$,当$|y-y_{0}|<\frac{\epsilon}{2L}$时,有$|f(x,y)-f(x,y_{0})|\le L|y-y_{0}|<\frac{\epsilon}{2}$。\\
取$\delta = \min\left\{\delta',\frac{\epsilon}{2L}\right\}$,则对于$\forall (x,y)\in G$,当$|x-x_{0}|<\delta$,且$|y-y_{0}|<\delta$时,有$|f(x,y)-f(x_{0},y_{0})|\le|f(x,y)-f(x,y_{0})|+|f(x,y_{0})-f(x_{0},y_{0})|<\epsilon$,即$f(x,y)$在$G$内连续。 
\end{frame}
\begin{frame}
    \frametitle{习题8} 
特别注意,这样的证法:\\$|f(x,y)-f(x_{0},y_{0})|\le|f(x,y)-f(x_{0},y)|+|f(x_{0},y)-f(x_{0},y_{0})|<\epsilon$是错误的,原因是二元函数$f(x,y)$在某区域$G$内对变量$x$连续是建立在固定$y$的基础上的。即用$f(x,y)$在$G$内对$x$连续的条件是无法说明对于$\forall \epsilon >0,\exists \delta'>0$,当$(x,y)\in G$,且$|x-x_{0}|<\delta'$时,有$|f(x,y)-f(x_{0},y)|<\frac{\epsilon}{2}$。(顺便说一下,这是一道18级计科的期中试题,一半的人使用了上述的错误证法)
\end{frame}
 

\begin{frame}[standout]
谢谢大家!
\end{frame}

 

\end{document}