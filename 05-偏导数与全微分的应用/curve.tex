% curve
% 空间曲线的切线与法平面
\section{空间曲线的切线与法平面}
\subsection{定义}
\begin{frame}
    \frametitle{定义}

    \begin{figure}
        \centering
        \def\svgwidth{.5\columnwidth}
        
        \scalebox{.6}{\input{img/tangent.eps_tex}}
        
        \caption{曲线$L$、割线$\overline{PQ}$和切线$\overline{PT}$\label{fig:tangent}}
    \end{figure}

    图~\ref{fig:tangent}当中曲线$L$上一点$P$处的切线$\overline{PT}$定义为:\alert{过$P$点的割线$\overline{PQ}$当$Q$点沿曲线趋于$P$的极限位置时的直线}。

\end{frame}

\subsection{切向量求法}
\begin{frame}
    \frametitle{切向量求法}

    $L$的曲线方程给出的形式不同,对于割线的计算也存在不同方法,可分为三种情况。

\end{frame}

\begin{frame}
    \frametitle{切向量求法·情形一}

    \begin{center}
        \fbox{\textbf{$L$的曲线方程为$x=x(t),\ y=y(t),\ z=z(t).\ \alpha\leq t\leq\beta$}}
    \end{center}\vfill\pause

    曲线上点$P(x_0,y_0,z_0)$为$t=t_0$时对应于$L$上的$P$点,即其坐标为$\left(x_0,y_0,z_0\right)$。$Q$点坐标为$\left(x(t),y(t),z(t)\right)$,则向量$\overrightarrow{PQ}$为$\left(x(t)−x_0,y(t)−y_0,z(t)−z_0\right)$,$\overrightarrow{PQ}$向量必然平行于向量
    \begin{equation}
        \left(\frac{x(t)-x_0}{t-t_0},\frac{y(t)-y_0}{t-t_0},\frac{z(t)-z_0}{t-t_0}\right)\label{eq:secant_vec}
    \end{equation}\vfill

\end{frame}

\begin{frame}
    \frametitle{切向量求法·情形一}
    令$t\to t_0$取极限则此时$\overline{PQ}$成为割线,从而割线上的向量成为切向量。所以$\overrightarrow{PQ}$成为切向量,则(\ref{eq:secant_vec})必然是切向量且$t\to t_0$取极限。切向量可以表示为\pause
    \begin{equation}
        \alert{\boldsymbol{\tau} = \pm\left(x^\prime(t_0),y^\prime(t_0),z^\prime(t_0)\right)}\label{eq:tangent_vec1}
    \end{equation}

\end{frame}

\begin{frame}
    \frametitle{切向量求法·情形二}

    \begin{center}
        \fbox{\textbf{$L$的曲线方程为$y=y(x),\ z=z(x)$}}
    \end{center}\vfill\pause

    可以将$x$视作$t$,$x=x,\ y=y(x),\ z=z(x)$的切向量变为\pause
    \begin{equation}
        \alert{\boldsymbol{\tau} = \pm\left(1,y^\prime(x_0),z^\prime(x_0)\right)}\label{eq:tangent_vec2}
    \end{equation}\vfill

\end{frame}

\begin{frame}
    \frametitle{切向量求法·情形三}

    \begin{center}
        \fbox{\textbf{$L$是两个曲面$F(x,y,z)=0,\ G(x,y,z)=0$的交线}}
    \end{center}\vfill\pause

    从$F$和$G$这个两个曲面方程中,必然存在$y$和$z$分别关于$x$的隐函数。设隐函数$y=y(x),\ z=z(x)$,切向量的形式仍然为$\boldsymbol{\tau} = \pm\left(1,y^\prime(x_0),z^\prime(x_0)\right)$。\pause

    求出$y$和$z$这两个隐函数关于$x$的导数,即对$F(x,y,z)=0$和$G(x,y,z)=0$两个方程关于$x$求导得到\pause
    \begin{equation}
        \begin{cases}
            F_x+F_yy^\prime+F_zz^\prime = 0 \\
            G_x+G_yy^\prime+G_zz^\prime = 0
        \end{cases}\Rightarrow%
        \begin{cases}
            F_yy^\prime+F_zz^\prime = -F_x \\
            G_yy^\prime+G_zz^\prime = -G_x
        \end{cases}\label{eq:hid_eq_der}
    \end{equation}\vfill

\end{frame}

\begin{frame}
    \frametitle{切向量求法·情形三}

    根据线性代数多元一次方程组的克拉默法则知识进行求解得到\pause
    \begin{equation}
        y^\prime%
        =\frac{\begin{vmatrix}-F_x&F_z\\-G_x&G_z\end{vmatrix}}{\begin{vmatrix}F_y&F_z\\G_y&G_z\end{vmatrix}}%
        =\frac{\begin{vmatrix}F_z&F_x\\G_z&G_x\end{vmatrix}}{\begin{vmatrix}F_y&F_z\\G_y&G_z\end{vmatrix}}%
        =\frac{\frac{\partial(F,G)}{\partial(z,x)}}{\frac{\partial(F,G)}{\partial(y,z)}}%
        \label{eq:hid_eq_der_solve_y}
    \end{equation}\pause
    \begin{equation}
        z^\prime%
        =\frac{\begin{vmatrix}F_y&-F_x\\G_y&-G_x\end{vmatrix}}{\begin{vmatrix}F_y&F_z\\G_y&G_z\end{vmatrix}}%
        =\frac{\begin{vmatrix}F_x&F_y\\G_x&G_y\end{vmatrix}}{\begin{vmatrix}F_y&F_z\\G_y&G_z\end{vmatrix}}%
        =\frac{\frac{\partial(F,G)}{\partial(x,y)}}{\frac{\partial(F,G)}{\partial(y,z)}}%
        \label{eq:hid_eq_der_solve_z}
    \end{equation}

\end{frame}

\begin{frame}
    \frametitle{切向量求法·情形三}

    所以得到切向量为\pause
    \begin{equation}
        \alert{\boldsymbol{\tau}=\pm\left(1,\frac{\frac{\partial(F,G)}{\partial(z,x)}}{\frac{\partial(F,G)}{\partial(y,z)}},\frac{\frac{\partial(F,G)}{\partial(x,y)}}{\frac{\partial(F,G)}{\partial(y,z)}}\right)}%
        \label{eq:tangent_vec3a}
    \end{equation}\pause
    也可以写成其平行向量形式\pause
    \begin{equation}
        \alert{\boldsymbol{\tau}=\pm\left(\frac{\partial(F,G)}{\partial(y,z)},\frac{\partial(F,G)}{\partial(z,x)},\frac{\partial(F,G)}{\partial(x,y)}\right)}%
        \label{eq:tangent_vec3b}
    \end{equation}

\end{frame}

\subsection{切线}
\begin{frame}
    \frametitle{切线}

    三种情形均可得到某点处切向量为$\boldsymbol{\tau}(x_0,y_0,z_0)\triangleq\pm(A,B,C)$\pause

    根据切向量得出切线方程为\pause
    \begin{equation}
        \alert{\frac{x-x_0}{A}=\frac{y-y_0}{B}=\frac{z-z_0}{C}}%
        \label{eq:tangent_line}
    \end{equation}\pause
    上述方程为三维空间直线的\alert{点向式}(\alert{对称式})方程。

\end{frame}

\subsection{法平面}
\begin{frame}
    \frametitle{法平面}

    $P$是切点。设点$P_1(x,y,z)$为法平面上一点,则向量$\overrightarrow{P_1P}$必然与切向量垂直。\pause
    \begin{equation}
        \alert{A(x−x_0)+B(y-y_0)+C(z-z_0)=0}%
        \label{eq:normal_plane}
    \end{equation}\pause
    满足(\ref{eq:normal_plane})的$(x,y,z)$必然在法平面上。

\end{frame}

\subsection{习题}
\begin{frame}
    \frametitle{习题一}

    求下列曲线在所示点处的切线方程和法平面方程:
    $$
        x=a\sin^2t,\ y=b\sin t\cos t,\ z=c\cos^2t,\ \text{在点}t=\frac{\pi}{4}
    $$

\end{frame}

\begin{frame}
    \frametitle{习题一·解答}

    $t=\frac{\pi}{4}$,对应的点为$\left(\frac{a}{2},\frac{b}{2},\frac{c}{2}\right)$。\pause
    
    因而可以计算得到\pause
    \begin{gather*}
        x^\prime\left(\frac{\pi}{4}\right)=\left.2a\sin t\cos t\right|_{t=\frac{\pi}{4}}=a\\
        y^\prime\left(\frac{\pi}{4}\right)=\left.b\left(\cos^2t-\sin^2t\right)\right|_{t=\frac{\pi}{4}}=0\\
        z^\prime\left(\frac{\pi}{4}\right)=\left.-2c\sin t\cos t\right|_{t=\frac{\pi}{4}}=-c
    \end{gather*}\pause
    代入(\ref{eq:tangent_vec1})和(\ref{eq:tangent_line})得到切线方程为
    $$\frac{x-\frac{a}{2}}{a}=\frac{y-\frac{b}{2}}{0}=\frac{z-\frac{c}{2}}{-c}$$\pause
    代入(\ref{eq:normal_plane})得到法平面方程为
    $$ax-cz=\frac{1}{2}(a^2-c^2)$$
    \qed

\end{frame}

\begin{frame}
    \frametitle{习题二}

    求下列曲线在所示点处的切线方程和法平面方程:
    $$
        2x^2+3y^2+z^2=9,\ z^2=3x^2+y^2,\ \text{在点}(1,-1,2)
    $$

\end{frame}

\begin{frame}
    \frametitle{习题二·解答}

    令$F(x,y,z)=2x^2+3y^2+z^2-9,\ G(x,y,z)=3x^2+y^2-z^2$,这时曲线方程为
    $$\begin{cases}
        F(x,y,z)=0\\
        G(x,y,z)=0
    \end{cases}$$\pause
    由%
    $\left(\begin{smallmatrix}
        F_x & F_y & F_z\\
        G_x & G_y & G_z
    \end{smallmatrix}\right)=\left(\begin{smallmatrix}
        4x & 6y & 2z\\
        6x & 2y & -2z
    \end{smallmatrix}\right)$知\pause
    \begin{gather*}
        \frac{\partial(F,G)}{\partial(y,z)}=\begin{vmatrix}6y&2z\\2y&-2z\end{vmatrix}=-16yz\\
        \frac{\partial(F,G)}{\partial(z,x)}=\begin{vmatrix}2z&4x\\-2z&6x\end{vmatrix}=20xz\\
        \frac{\partial(F,G)}{\partial(x,y)}=\begin{vmatrix}4x&6y\\6x&2y\end{vmatrix}=-28xy
    \end{gather*}

\end{frame}

\begin{frame}
    \frametitle{习题二·解答}

    代入(\ref{eq:tangent_vec3b})得到切向量为
    $$\boldsymbol{\tau}=\left.(-16yz,20xz,-28xy)\right|_{(1,-1,2)}=4\cdot(8,10,7)$$\pause
    代入(\ref{eq:tangent_line})得到切线方程为
    $$\frac{x-1}{8}=\frac{y+1}{10}=\frac{z-2}{7}$$\pause
    代入(\ref{eq:normal_plane})并整理得到法平面方程为
    $$8x+10y+7z=12$$
    \qed

\end{frame}