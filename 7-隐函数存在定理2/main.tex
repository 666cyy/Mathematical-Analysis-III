%!TEX TS-program=xelatex
\documentclass[xetex]{beamer}
% 将上面这一行修改成下面这个样子,可以创建适合于发布的版本,这去除了所有的动画
%\documentclass[xetex, handout]{beamer}

% 规范注意:

% 使用正确的主题(beamer主题、文字字体)
% 使用正确的title信息(title、subtitle、author、date)
% 合理使用frame 和 standout frame
% 块(block、exampleblock、alertblock)以及文字段落内部的\alert的使用
% 使用图片需要使用\begin{figure}...\end{figure}并附带\caption和\label信息
% 使用enumerate和itemize组织你的点
% 使用section给你的幻灯片分部分
% 公式,文字段落内嵌公式和单独的公式块的使用
% \DeclareMathOperator的使用,以及学会在网上查找你不知道怎么输入的数学符号
% 动画的使用
% 讲课录制使用什么版本的文档;对外发布使用什么版本的文档(handout)

\usefonttheme{professionalfonts}

\usepackage[UTF8]{ctex}
\usepackage{hyperref}
\usepackage{unicode-math}
\usepackage{amsmath, amssymb}
\usepackage{graphicx, wrapfig}

\usepackage{nopageno}

\DeclareMathOperator{\argmax}{argmax}

\usetheme[block=fill]{metropolis}

\setmathfont{XITS Math}

\title{隐函数存在定理II}
\author{数学分析MOOC小组}
\date{}

\begin{document}

\frame{\maketitle}

\begin{frame}
    \frametitle{方程组确定的隐函数组}
    方程组$
    \begin{matrix}
    F(x,y,u,v)=0\\
    G(x,y,u,v)=0
    \end{matrix}
	$\\
    在什么情况下,能确定两个变量为另外两个变量的函数的隐函数组呢?不妨以确定$u=u(x,y)\quad v=v(x,y)$为例\\
    下面是方程组的隐函数组存在定理的内容:
    对于一个点$P_0=(x_0,y_0,u_0,v_0)$\\
    \begin{enumerate}
    	\item $F(P_0)=0$\quad $G(P_0)=0$
    	\item $P_0$附近的邻域内\\
    	$F(x,y,u,v),G(x,y,u,v)$的各个一阶偏导数连续\\
    	\item 与单个方程不同的情形是,这里需要满足$J|_{P_0}=\frac{\partial (F,G)}{\partial (u,v)}|_{P_0}=
    	\left|\begin{matrix}
    		F_u&F_v\\
    		G_u&G_v
    	\end{matrix}\right|
    	\not =0$\\
    	
    \end{enumerate}
    
\end{frame}

\begin{frame}
\frametitle{方程组确定的隐函数组}
	为了方便记忆,大家可以记住\\
	被确定为隐函数(组)因变量的变量,例如这里的$u,v$\\
	原函数$F,G$的偏导构成的Jacobi行列式\\
	在$P_0$点处必须非0 \\ \pause
	然后我们来证明如何从这个条件得到隐函数存在\\
	由于课本已经有完整的过程,我这里只是再提炼一下大概的流程\\
	由条件3\quad$\left|\begin{matrix}
		F_u&F_v\\
		G_u&G_v
	\end{matrix}\right|$在$P_0$处$\not =0$\quad $F_u|_{P_0}$和$F_v|_{P_0}$至少有一个$\not =0$\\ \pause
	不妨设$F_v|_{P_0}\not =0$,由单个方程的隐函数存在定理\\
	在点$P_0$的邻域附近,方程$F(x,y,u,v)=0$唯一确定隐函数$v=\phi(x,y,u)$\\ \pause
	$v_0=\phi(x_0,y_0,u_0),F(x,y,u,\phi(x,y,u))\overline{=}0$\\
	记$\psi(x,y,u)=G(x,y,u,\phi(x,y,u))$\\
	方程组$
	\begin{matrix}
		\psi(x,y,u)=G(x,y,u,\phi(x,y,u))\quad (1)\\
		F(x,y,u,\phi(x,y,u))=0	\quad (2)
	\end{matrix}
	$
\end{frame}

\begin{frame}
\frametitle{方程组确定的隐函数组}
	对(1)式左右两边对u求偏导\\
	$\psi_u=G_u+G_v\phi_u$\quad $F_u+F_v\phi_u=0$ \\ \pause
	所以$\psi_u=G_u-G_v\frac{F_u}{F_v}=\frac{G_uF_v-G_vF_u}{F_v}=-\frac{1}{F_v}\left|\begin{matrix}
		F_u&F_v\\
		G_u&G_v
	\end{matrix}\right|=-\frac{1}{F_v}\frac{\partial (F,G)}{\partial (u,v)}$\\ \pause
	因为$F_v(P_0)\not=0,\frac{\partial (F,G)}{\partial (u,v)}$在$P_0$处$\not=0$\\
	所以$\psi_u(x_0,y_0,u_0)\not=0$\\ \pause
	所以$\psi(x,y,u)=G(x,y,u,\phi(x,y,u))=0$满足隐函数存在定理\\
	因此在$P_0$的邻域内,$\psi(x,y,u)=0$唯一确定隐函数$u=u(x,y)$\\
	所以$v=\phi(x,y,u)=\phi(x,y,u(x,y))=v(x,y)$\\
	综上所述,方程组$F(x,y,u,v)=0,G(x,y,u,v)=0$\\
	在给定的条件1,2,3下\quad 可以唯一确定隐函数组$u=u(x,y),v=v(x,y)$\\
	
	
\end{frame}


\begin{frame}
\frametitle{方程组确定的隐函数组}
	对于隐函数组$u=u(x,y),v=v(x,y)$的偏导数$\frac{\partial u}{\partial x},\frac{\partial u}{\partial y},\frac{\partial v}{\partial x},\frac{\partial v}{\partial y}$\\
	通过对方程组$F(x,y,u,v)=0,G(x,y,u,v)=0$
	\\两边对$x,y$求导得到,也就是\\
	$F_x+F_u\frac{\partial u}{\partial x}+F_v\frac{\partial v}{\partial x}=0$\quad $F_y+F_u\frac{\partial u}{\partial y}+F_v\frac{\partial v}{\partial y}=0$\\
	$G_x+G_u\frac{\partial u}{\partial x}+G_v\frac{\partial v}{\partial x}=0$\quad $G_y+G_u\frac{\partial u}{\partial y}+G_v\frac{\partial v}{\partial y}=0$\\
	所以$$\left[
		\begin{matrix}
			F_u & F_v\\
			G_u & G_v
		\end{matrix}
	\right]\left(
		\begin{matrix}
			\frac{\partial u}{\partial x}\\
			\frac{\partial v}{\partial x}\\
		\end{matrix}
	\right)=-\left(
		\begin{matrix}
			F_x\\
			G_x
		\end{matrix}
	\right)\quad
	\left[
	\begin{matrix}
	F_u & F_v\\
	G_u & G_v
	\end{matrix}
	\right]\left(
	\begin{matrix}
	\frac{\partial u}{\partial y}\\
	\frac{\partial v}{\partial y}\\
	\end{matrix}
	\right)=-\left(
	\begin{matrix}
	F_y\\
	G_y
	\end{matrix}
	\right)
	$$
	答案见课本的P233页,下面我们带来有关例题的讲解\\

	
\end{frame}


\begin{frame}
\frametitle{方程组确定的隐函数组}
	P239 6.
	设$$
	\begin{matrix}
		u=f(x,y,z,t)\\
		g(y,z,t)=0\\
		h(z,t)=0
	\end{matrix}
	$$\\
	在什么条件下,$u$是$x,y$的函数,求$\frac{\partial u}{\partial x},\frac{\partial u}{\partial y}$\\ \pause
	我们观察这个方程组,可以看到有5个变量$x,y,z,t,u$\\ 
	不妨设方程$h(z,t)=0$,可以确定$z$是$t$的隐函数,设为$z=\phi(t)$\\ \pause
	对应的导数满足$\frac{dz}{dt}=-\frac{h_t}{h_z}$\\ \pause
	考虑方程$g(y,z,t)=0$,也就是$g(y,\phi(t),t)=0$\\
	可以确定$t$是$y$的隐函数,设为$t=\psi(y)$\\
	方程2对$y$求导得到,$g_y+g_z\frac{dz}{dt}\frac{dt}{dy}+g_t\frac{dt}{dy}=0$\\ \pause
	也就是$(\frac{g_t h_z-g_z h_t}{h_z})\frac{dt}{dy}=-g_y$ \quad
	$\frac{dt}{dy}=\frac{g_y h_z}{g_z h_t-g_t h_z}=\frac{g_y h_z}{
	\left|
		\begin{matrix}
			g_z & g_t\\
			h_z & h_t
		\end{matrix}
	\right|
	}=\frac{g_y h_z}{\frac{\partial (g,h)}{\partial (z,t)}}$\\

	
\end{frame}

\begin{frame}
\frametitle{方程组确定的隐函数组}
	把这两个隐函数代回$u=f(x,y,z,t)$\\
	可以得到$u=f(x,y,\phi(\psi(y)),\psi(y))$\\ \pause
	所以此时$u$是关于$x,y$的函数\\
	$\frac{\partial u}{\partial y}=f_x$\\
	$\frac{\partial u}{\partial y}=f_y+f_z\frac{dz}{dt}\frac{dt}{dy}+f_t\frac{dt}{dy}=f_y-f_z\frac{h_t}{h_z}\frac{g_y h_z}{\frac{\partial (g,h)}{\partial (z,t)}}+f_t\frac{g_y h_z}{\frac{\partial (g,h)}{\partial (z,t)}}$\\ \pause
	$\quad = f_y+\frac{g_y h_z}{\frac{\partial (g,h)}{\partial (z,t)}}\frac{f_t h_z-f_z h_t}{h_z}=f_y-\frac{g_y
			\left|
		\begin{matrix}
		f_z & f_t\\
		h_z & h_t
		\end{matrix}
		\right|
	}{\frac{\partial (g,h)}{\partial (z,t)}}=f_y-g_y\frac{\frac{\partial (f,h)}{\partial (z,t)}}{\frac{\partial (g,h)}{\partial (z,t)}}$\\ \pause
	由于要满足两个隐函数存在定理\\
	因此满足条件为:
	\begin{enumerate}
		\item
			$f,g,h$三个函数均连续并有一阶连续偏导数
		\item
			存在点$P(x_0,y_0,z_0,t_0,u_0)$\quad 使得方程组在点P处成立\\
		\item
			$h_z(z_0,t_0)\not =0\quad \frac{\partial (g,h)}{\partial (z,t)}(y_0,z_0,t_0)\not =0$\\
			
	\end{enumerate}
\end{frame}

\begin{frame}[standout]
	谢谢
\end{frame}

\end{document}